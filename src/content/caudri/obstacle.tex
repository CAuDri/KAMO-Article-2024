\section{Zeitlupe}
Man kann in Längen beschreiben, was den Wettbewerb so schwer macht.
Ich denke aber Am besten geht das, indem wir die Zeit kurz einmal 
anhalten und uns im Detail anschauen, was eine der grundlegendsten 
Aufgaben alles an Schwierigkeiten beinhaltet...
In der Fahrschule wohl Teil der ersten Stunde, auch wenn man dort erstmal nicht so schnell fährt, wie wir: Die Kurve.

Wenn ein Auto um eine Kurve fährt, ist bereits jede Rolle im Team involviert.
Doch es beginnt natürlich erstmal mit der Wahrnehmung der Kurve.
Die Kamera erkennt erstmal einen $x times x$ Pixelmatrix.
Bei manchen Teams mit Tiefenkamera sogar eine Punktwolke. 
Nun steigt die Software des Teams ein und erkennt - auf irgendeine Weise -
dass die Fahrbahn nun eine Rechtskurve macht. 
Je nachdem, welches Streckenelement vor der Kurve kam muss diese Verarbeitung auch 
sehr schnell geschehen. Denn wenn davor eine Gerade lag ist das Auto sehr schnell,
wenn es gerade ein Hindernis umfahren hat, sieht es die vorrausliegende Strecek erst
unmittelbar, bevor sie befahren wird. 
Wenn ein Team die Bilderkennung mit Neuronalen Netzen macht, müssen sie biespielsweise
oft darauf achten, dass das Netz schnell genug läuft.
Generell scheitern bei der Bilderkennung viele coole Ideen daran, dass sie 
letzten Endes doch einige Millisekunden zu langsam sind. 

Nachdem das Auto erkannt hat, dass die Fahrbahn eine Biege macht, übernimmt
in einer Form oder der anderen das Gehirn des Autos. 
Es vereinigt diese Informationen mit allen weiteren Merkmalen in der Umwelt,
die erkannt worden sind und mit dem Zustand, in dem sich das Auto gerade befindet. 
Sprich: Freie Fahrt auf einer Geraden, Kreuzung, Überholvorgang, Parken, etc.
Anschließend bringt es diese Informationen zusamemn und generiert die Trajektorie, sprich die Fahrt, die es
auf den nächsten Zentimetern umsetzen will. 

Nun setzt die Regelung des Autos ein, um die Motoren und Lenkung so 
anzusprechen, dass das Auto der Trajektorie optimal und möglichst schnell folgt.
Dafür muss das Auto zum Beispiel berechnen, wie schnell es fahren darf, ohne aus der Kurve zu fliegen.
Außerdem muss es den Lenkwinkel so variieren, dass die Kurve möglichst eng und mit hoher Geschwindigkeit passieren kann.
Allein diese beiden Dinge hängen bereits voneinander ab und müssen gemeinsam optimiert werden!

Schließlich hat dann die Hardware Abteilung bereits dafür gesorgt, dass das Auto möglichst stabil und mit möglichst hoher Haftung 
durch die Kurve fahren kann.

Je nachdem, was für ein Hindernis gerade bevorsteht, werden natürlich auch verschiedene Fähigkeiten in besonderem
Maße gefordert. Beispielsweise ist ein gerader Streckenabschnitt für die Wahrnehmung wohl keine große Herausforderung, durch die Konstruktion 
und optimale Ansteuerung des Autos kann hier aber sehr viel rausgeholt werden!
Umgekehrt ist es beispielsweise mit komplexen Kreuzungssituationen oder Kombinationen von Hindernissen und Streckenmarkierungen.
Besonders hier müssen Rechts-vor-Links, die Prioritäten und Bedeutungen von allen Dinge und co. richtig im Gehirn des Autos einprogrammiert sein.
