\section{Was bisher geschah}
Im Kontext von autonomen Fahrens waren Wettbewerbe und studentische Teilhabe auch in der Wiege des Feldes schon beteiligt.
Die DARPA (Defense Advanced Research Projects Agency) Grand Challenges in 2004 und 2005 des Amerikanischen Verteidigungsministeriums
sind hierfür wohl das erste Beispiel. 
Hier mussten die Teilnehmer eine 240 Kilometer lange Autobahnstrecke entlangfahren. In 2004 schaffte das noch keiner, in 2005 dann schon
fast alle. Wenn man sich diesen Fortschritt innerhalb eines Jahres anschaut, erwartet viel für die beiden folgenden Jahrzehnte.

In der Tat hat, wie Leser bestimmt auch mitbekommen hat, vor über einem Jahr Mercedes Benz
das erste SAE (Society of Automotive Engineers) Level 3 konforme Fahrzeug in den USA zugelassen.
Es hat inzwischen also auch hundertausende an Endverbrauchern erreicht, die auch während der Fahrt nicht 
am Verkehr teilnehmen müssen und sich darauf verlassen können, dass ihr Fahrzeug ihnen die Kontrolle ruhig und
sicher wieder überlassen kann.

Allerdings liegt das Ziel natürlich noch höher und so ist das Thema auch heute noch relevant wie eh und jeh.
Gleichzeitig werden in Zukunft angrenzende Themen weiter in den Vordergrund rücken. Zum Beispiel das Vernetzen von Infrastruktur 
und Fahrzeugflotten kommt hier in den Sinn.

Als Studenten wollen wir Teil dieser Entwicklungen sein.
Ungefähr ein Jahr nach der letzten DARPA Grand Challenge startete ein weiterer Wettbewerb für autonomes Fahren in die erste Runde.
Der Carolo Cup hatte sehr ähnliche Regeln wie unsere CAuDri Challenge und fand bis 2021 jährlich statt.
Auch einige unserer Teams nahmen daran bis zu dessen Ende teil. 
Mit bis zu 20 teilnehmenden Teams katalsysierte der Cup zum einen die Entwickler von vielen autonom fahrenden Modellautos.
Vor allem bat er aber diesen 20 Teams und dadurch hunderten an involvierten Studenten die Möglichkeit, selbstständig
an großen und komplexen Projekten und Teams zu arbeiten und Teil dieses faszinierenden Gebiets zu sein.

Ende 2022 gab es dann ein Treffen zwischen KITcar e.V. aus Karlsruhe, Team Spatzenhirn aus Ulm und Team Smart Rollerz aus Stuttgart.
Neben dem freundschaftlichen Austausch und gegenseitigen Vorführen der jeweiligen Autos wurde dort auch beschlossen sich in größerem,
offizielleren Rahmen zu treffen und weitere Teams, die wir noch in unseren Kontaktlisten hatten, einzuladen. 
Und so fand im September 2023 die erste Iteration der CAuDri Challenge statt und am 21. und 22. Juli 2024 steht die zweite Iteration
wieder mit Unterstützung der DHBW Stuttgart in deren Räumlichkeiten statt. Somit wollen wir auch in Zukunft uns und weiteren Studernten die
Möglichkeit geben, sich in völliger Autonomie technisch ausleben zu können und an Themen des auonohmen Fahrens zu experimentieren und zu forschen.
