\section{Unsere Pläne}
Es gibt zwei konkrete Punkte, die wir in Zukunft - abgesehen 
davon, mehr Teilnehmerteams, sowie Sponsoren mit ins Boot zu holen -
erkunden möchten. 
Der eine sind die Form der Disziplinen und der andere die Einstiegshürde
für neue Teams zu senken.

\subsection{Disziplinen}
Wir sind mit der Form unserer derzeitigen 2 Disziplinen nur teilweise zufrieden.
Ich möchte an dieser Stelle zuerst die Details im Regelwerk
ansprechen, die uns stören.
Denn wir wissen selber nicht, ob wir uns nun über den Aspekt "Rennenfahren"
oder über den Aspekt "Straßenverkehr" identifizieren. 
Der Free Ride gehört natürlich zu ersterem, der Obstacle Course zu letztem.
In den letzten Jahren war sogar der Obstacle Course noch auf Zeit.
In diesem Jahr haben wir dies abgeschafft, um die beiden Modi weiter abzugrenzen.
Insgesamt wollen wir in den kommenden Jahren darauf achten, dass man in den
beiden Disziplinen und in unserem gesamten Wettbewerb klar abgegrenzte Schwerpunkte 
erkennen kann.

Nun möchte ich noch einen Schritt zurückgehen. 
Denn während es für uns Teams sehr schwierige und spannende Herausforderungen sind, 
haben wir das Gefühl, dass vor allem in Zukunft andere
Themen relevanter sein werden. 
Zum Beispiel an den Themen in diesem Magazin kann man sehen, dass die Fahrbah- und
Hindernisnerkennung nichtmehr die Probleme sind, die die Forschung und Industrie
vor Schwierigkeiten stellen. 
Gleichzeitig bringt es uns größeren Reiz das Gefühl zu haben an auch in der heutigen 
noch sehr relevanten und wichtigen Themen mitzuarbeiten. 

In der ersten Gesprächen, die wir dazu geführt haben kam als wichtigstes Thema 
das Fahren im vernetzten Verbund auf. Gerade hier kann man sich gut vorstellen,
wie wir einmal im kleinen Konzepte demonstrieren oder neue Standards und Protokolle
als Pioniere im kleinen Maßstab ausprobieren könnten. Man könnte die Anfänge 
durch erste vernetzte Infrastruktur, zum Beispiel Ampeln anstellen, bis hin zu
irgendwann mehrere (in dem Fall Kollisionsresistente!) Fahrzeuge gleichzeitig 
fahren zu lassen.

Spannend wäre es sicherlich auch, wenn wir in unserem Wettbewerb die
Security und Safety Fragen in spannender Weise abbilden könnten. Hier
haben wir allerdings leider viel weniger Ideen, als davor.

Bei diesen Überlegungen ist aber auch noch offen, in welcher Form ein neues
Thema den Weg in die CAuDri-Challenge finden würde. 
An dieser Stelle muss man auch das zweite Problem noch berücksichtigen:
Die Einstiegshürde ist im Moment noch zu hoch.
Die Problemstellung ist sehr schwer. KITcar e.V. entwickelt beispielsweise
nun seit 10 Jahren an ihrem Auto und Software.
Dieser Vorsprung ist kaum einzuholen.

Unmöglich ist es bei weitem nicht; schließlich hat sich gerade in diesem
Bereich in den letzten 10 Jahren soviel getan und entwickelt, dass 
es zu dem Großteil der Teilprobleme inzwischen auch vorgefertigte 
ROS Pakete im Internet gibt. 
Nichts desto trotz sind wir sehr daran interessiert, es für neue Teams
so einfach und gleichzeitig aber so fair wie möglich zu machen, neu 
beim Wettbewerb einzusteigen. 
Zu diesem Zweck wollen wir sowohl an der Bewertung und Wettbewerbsstruktur,
als auch mit der Herausgabe einer "Starthilfe" ansetzen.

\subsection{Fliegender Start}
Ein einfacher Weg wäre an der Bewertung anzusetzen und das eventuell mit 
einem neuen Theam zu kombinieren.
So könnten wir zum Beispiel in einem Jahr 
ein neues Thema, bzw. eine neue Disziplin für das kommende ankündigen und für
diese eine unabhängige Wertung einführen.
Dann müssten neue Teams zwar von null anfangen, die anderen aber annähernd auch.
Natürlich wäre das als etabliertes Team immernoch viel einfacher, aber umso
größer wäre der Eindruck wenn ein neues Team alle altbekannten Teams abhängen könnte.
Schließlich müssten diese ihre Mühen auf alle 3 Disziplinen verteilen, während
sich ein neues Team erstmal nur auf eins konzentrieren könnte.

Ein ähnliches Modell wäre es, einen Hackathon an einem der Wettbewerbstage zu 
veranstalten. Dann wäre selbst für Teams etwas geboten, die noch gar kein Auto
fertiggebaut haben. Dafür war die Idee bisher, dass sich unabhängig der großen Teams
z.b. vierer Teams bilden können, die alle von Grund auf eine Aufgabe lösen müssen
ohne das Auto ihrers usprünglichen Teilnehmerteams nutzen zu dürfen.

An dieser Stelle möchten wir darauf hinweisen, dass wir uns hier auch über Ideen
und Anregungen freuen würden. 
Falls Sie coole Ideen haben oder eine besondere 
bezüglich, welche Theemn in der Zukunft relevant werden und wie man diese umsetzen könnte,
nehmen Sie gerne Kontakt mit uns auf!

Das größte, aber wohl auch effektivste Projekt, dass wir zu dem Zweck angehen werden,
läuft aber unter dem Codename "LARS" - <Hier Acronym einfügen>.
Das soll in erster Iteration ein Auto darstellen, das einfach und günstig zu bauen ist,
aber bereits für die grundlegenden Aufgaben ausreicht und gut erweiterbar ist.
Wir sind noch auf der Suche nach Unterstützung im Design von LARS. 
Schließlich wollen wir eine Anleitung veröffentlichen mit den genauen
Bauteilen, die benutzt werden, der man ohne große Vorkenntnisse 
folgen kann.

Anschließend kann man sich auch überlegen auch Code mit zu veröffentlichen.
Vielleicht würde es auch schon reichen, wenn alle Teams ihre Datensätze zusammentragen 
und diese veröffentlichen. Beziehungsweise einer Mischung dieser beiden Ideen wäre,
dass der Gewinner der Challenge seinen Code veröffentlichen muss.
Auch diese Regel gefällt uns zu großen Telien gut.

So oder so bleibt für uns zu klären, wie wir dieses Starterkit entwickeln wollen.
Bisher stellen wir uns einen dieser Ansätze vor.
Entweder, wir überlassen es eines der Teilnehmerteams basierend auf ihrem Auto ein 
günstiges und simpleres Modell herauszudestillieren. 
Oder wir bilden eine teamübergreifende Arbeitsgruppe, die aus allen Teams
Ideen einfließen lässt.
So oder so würde es Kapazität von den Teams einfordern, die dann nicht in die 
Hauptautos fließen, die bei der Challenge teilnehmen.
Alternativ wäre es denkbar, dass wir einen Partner finden, der uns entweder aktiv
in der Entwicklung unterstützt oder bei dem wir einen Studenten von uns anstellen
können, um das Auto zu entwickeln.
