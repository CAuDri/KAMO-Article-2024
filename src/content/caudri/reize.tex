\section{Warum ist das cool?}
Aber genug davon, was die CAuDri Challenge genau ist. 
Der Leser stellt sich jetzt bestimmt die Frage, warum wir uns als Studenten überhaut die Mühe machen, zusätzlich zur technischen Entwicklung auch noch die Challenge zu organisieren.
Schließlich gibt es auch alternative Wettbewerbe.

\subsection{Weil auch der Weg ein Ziel braucht!}
Allerdings hängt die Challenge unmittelbar mit uns teilnehmenden 
Hochschulgruppen zusammen.
Denn ohne den Wettbewerb hätten wir keine unmittelbare Motivationsgrundlage mehr.
Schließlich richten wir unsere gesamte Entwicklung danach, am Stichtag 
möglichst gut zu performen und die Aufgaben des Regelwerwks zu lösen.
In keiner anderen Weise hätten wir so viel Freiheit und könnten an so coolen Projekten 
arbeiten, wie mit der CAuDri-Challenge.

Somit ist der Wettbewerb essentiell für die Teams, die bereits teilnehmen.
Gleichzeitig - so hoffen wir es zumindest - bieten wir dadurch auch neuen Teams
die Möglichkeit, sich so mit coolen Teamen zu befassen und direkt etwas zu haben,
worauf sie hinarbeiten können. Zukünftig wollen wir auch den Einstieg durch Tutorials
und Starthilfen erleichtern, um dadurch auch die Konkurrenz und das Ökosystem zu erweitern.


\subsection{Ein Spielplatz}
Die Vorteile für uns Studenten liegen abewr auf der Hand. Wir bekommen durch 
die Clubs die Möglichkeit, Erfahrungen zu sammeln und finden an unseren Unis Gemeinschaften
auch unabhängig von den technischen Fasetten. 

Technisch könenn wir uns in diversen Themen austoben.
Von Informatik über Machine Learning und Robotik/ROS bis hin zu
Embedded Systems, Fahrzeug-, Antriebs-, Regelungs- und Elektrotechnik
ist alles dabei. Das wäre alleine auf der Größenordnung umöglich.

Auch im Vergleich zu Anstellungen an der Uni oder der Industrie stechen solche
Gruppen heraus, da wir hier eben in kompletter Eigenrechie arbeiten und über
die ganze Prozess- und Organisationskette hinweg die Verantwortung dafür tragen, 
dass alle optimal Zusammenarbeiten und unser Auto am Ende ein gutes Ergebnis abliefern kann!
Dadurch, dass unsere Teams in der Reegl aus Studenten bestehen, sind die meisten Clubs
auch davon angewiesen, sich gegenseitig fortzubilden und ihr wissen jeweils 
an die nächste Generation weiterzugeben. 

Generell können wir uns in unseren Gruppen auch was die Organisation unserer
Arbeitsgruppen angeht austoben. Dabei muss dieser Wissenstransfer sichergestellt werden,
denn viele haben bei Ihrem Start vielleicht noch gar keine
Ahnung von den anstehenden Themen. Gleichzeitig sollte es jedem Spaß machen,
wenn die Gruppe auch mehrere Teilnahmen überstehen soll. 
Gerade bei den großen Hochschulgruppen, die an der Challenge teilnehmen ist diese
Seite mit bis zu 60 Mitgliedern nicht weniger wichtig, als der technische Fortschritt.\
Und auch wenn die Techniker auf diese Weise mal dazu gezwungen werden, sich
mit der Organisation und Zusammenarbeit des Teams auf höheren Ebenen auseinanderzusetzen.

Für wieder andere ist dieser Mix aus technischer und teamlicher Verantwortung
genau das richtige.

\subsection{Networking und Außendarstellung}
Für die Studenten ist ein solcher Wettbewerb natürlich auch die Möglichkeit Präsenz zu
zeigen und Partner auf sich aufmerksam zu besuchen. 
Besonders in den kommen Jahren planen wir damit, die Unterstützung von Sponsoren 
gewinnen zu können. 

Davon profitiert logischerweise sowohl der Wettbewerb, als auch die einzelnen Studenten,
die bei dem Event mit den Vertretern aus Industrie und Forschung in Kontakt treten können.

\subsection{Alleinstellungsmerkmale}
Die CAuDri-Challenge zeichnet aus, dass die Auto sehr komplexe Verkehrssituationen
bewälitgen müssen und dass die Autos von Grund auf eigen konzipiert und gebaut werden.
Bei ähnlichen Wettbewerben steht oft das Rennen im Fordergrund und es gibt keine Hindernisse
oder schwierige Situationen, in denen Entscheidungen getroffen werden müssen.

Parallel dazu ist auch besonders, dass wir die CAuDri-Challenge eben aus eigener
Hand heraus und als ehrenamtlicher Verein organisieren. Dadurch können wir sicherstellen,
dass die Veranstaltung immer in erster Linie zur Unterstützung von uns Arbeitsgruppen
sein wird.
