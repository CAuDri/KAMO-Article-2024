\section{Allgemein}
Im Allgemeinen ist die CAuDri-Challenge ein studentischer Hochschulwettbewerb, bei dem verschiedene
ehrenamtliche Gruppen von unterschiedlichen Hochschulen zusammenkommen. Der Wettbewerb wurde letztes Jahr 2023
zum ersten Mal in Stuttgart ausgetragen. Die Hochschulgruppen entwickeln in Ihrer Freizeit ein autonom Fahrendes
Modellfahrzeug im Maßstab 1:10 und treten an der Challenge in zwei verschiedenen Disziplinen gegeneinander an.

\section{Disziplinen} 
Während des gesamten Events wird die Performance der Modellfahrzeuge in zwei unterschiedlichen
Disziplinen auf die Probe gestellt und währenddessen von Schiedsrichtern bewertet. Die erste Disziplin ist der
„Freedrive“, wo das Fahrzeug in einer vorgegebenen Zeit so viele Streckenmeter wie möglich zurücklegen
muss. Die zweite Disziplin bezeichnet sich als „Obstacle Evasion“. Hier wird die Strecke um zusätzliche
Hindernisse erweitert, welche beim Fahren berücksichtigt werden müssen.

Wie bereits erwähnt wird die Performance des Fahrzeugs von Schiedsrichtern bewertet. Für jede Disziplin wird
eine bestimmte Anzahl von Punkten vergeben, welche sich aus verschiedenen Faktoren wie zum Beispiel der insgesamt
zurückgelegten Strecke und das erfolgreiche Passieren unterschiedlicher Hindernisse. Werden Regularien verletzt,
werden entsprechend Punkte abgezogen.  

\subsection{Freedrive} Die „Freedrive-Disziplin“ dient dazu die allgemeine
Fahrsituation eines außerörtlichen Fahrszenarios zu simulieren. Auf der gesamten ausgelegten Fahrstrecke befinden
sich demnach keine Hindernisse. Das Fahrzeug muss während dem Abfahren der Strecke nur Straßenkreuzungen beachten
und die entsprechenden Fahrbahnmarkierungen nicht überfahren.

Wie bereits oben erläutert wird in dieser Disziplin die Schnelligkeit des Fahrzeugs in Bezug auf die zurückgelegte
Strecke nach einer bestimmten Zeit bewertet. Bei der „Freedrive-Disziplin“ wird gleichzeitig das Parken
simuliert. Dazu werden an einer bestimmten Stelle verschiedene Parkbuchten eingebaut. Einige der Parkflächen
werden dabei mit einem Hindernis blockiert um dadurch einen bereits besetzten Parkplatz zu simulieren. Bei jedem
Streckendurchlauf soll das Fahrzeug in einer der freien Parkbuchten jeweils einen vollständigen Parkvorgang
(einmal ein- und wieder ausparken) durchführen.

Bei dieser Disziplin wird eine bestimmte Punkteanzahl für die insgesamt zurückgelegte Strecke vergeben. Zusätzliche
Punkte können durch erfolgreich durchgeführte Parkvorgänge erzielt werden.  

\subsection{Obstacle Evasion Course}
Die „Obstacle-Evasion-Disziplin“ dient der Simulation eines Innenstadtszenarios. Dabei werden auf der gesamten
Strecke an verschiedenen Stellen zusätzliche dynamische und statische Hindernisse eingebaut, die es bei der
Fahrt zu berücksichtigen gilt. Parkvorgänge werden in dieser Disziplin nicht durchgeführt. Zusätzlich zu den
statischen und dynamischen Hindernissen werden Straßenschilder, Geschwindigkeitsbegrenzungen sowie Vorfahrtsregeln
und Abbiegen entlang der Strecke eingebaut. Bei den statischen Hindernissen handelt es sich zumeist um weiße
Boxen welche mit einem festgelegten Abstand auf die Fahrbahn drapiert werden und welche das Fahrzeug dann umfahren
muss. Bei den dynamischen Hindernissen kann es sich zum einen um eine sich auf der Fahrbahn bewegende Box handeln,
welches ein vorausfahrendes Auto simulieren soll. Zum anderen kann es sich um einen Fußgänger handeln, der über
einen Zebrastreifen die Fahrbahn überquert.

Das Ziel dieser Disziplin ist es, die größte Distanz zurückzulegen und dabei die eingebauten und simulierten
Herausforderungen erfolgreich abzuschließen. Sollten die eingebauten Hindernisse nicht eingehalten oder missachtet
werden, gibt es dafür jeweils Strafen, welche sich in einem Punkteabzug in der Bewertung widerspiegeln.


\subsection{Das Auto bauen}
Die beiden Disziplinen lenken den Fokus auf die Herausforderungen, die die Software des Autos bewältigen muss.
Dabei darf man aber eines nicht vergessen; denn jedes Team muss ich Auto von Grund auf selbst konzipieren und konstruieren.
Die Rahmenbedingungen, die die Autos beachten müssen sind dabei sehr flexibel. 
Lediglich die Höhe der Autos und ihre allgemeine Bauform sind beschränkt.

Eine liebliche Konsequenz daraus ist, dass manche Autos mehr als solche wahrgenommen werden,
andere vielleicht eher als Kühlschrank mit Zweiachsenlenkung.
// TODO: Bilder von uns und Ulm einfügen
So oder so müssen unsere Teams dadurch natürlich eine viel größere Bandbreite an Diziplinen abdecken.

Denn sowohl über Herstellung der Bauteile, wie auch Planung der Elektronik
und das Zusammenspiel derer müssen die Teams alles selber machen. Dabei
gilt es, die Fahrdynamik zu optimieren, gleichzeitig eine Plattform zu kreieren,
die zb. in langen Test-Sessions gut benutzwar und variabel ist,
möglichst erweiterbar ist, um Dinge auszuprobieren, gut reparabel ist,
wenn der neue Praktikant das Gefährt wieder gegen die Wand gefahren hat
Denn die Anforderungen an die Autos sind hoch und  bilden 
Dadurch deckt die CAuDri-Challenge auch, aber natürlich auch
den Software Abteilungen möglichst gute Schnittstellen und möglichst
viel Freiheiten zum optimiern bietet.

Insgesamt wird die CAuDri-Challenge durch diese Hardware-Komponente
abwechslungsreicher und interessanter.
Es zwingt die Teams dazu, durch ihre Mitglieder eine größere Bandbreite
abzudecken und hebt die Chalenge von anderen Wettbewerben ab, die sich
auf die Software beschänken.

Darüber, wie diese Hardware Abteilung in den Teams aufgebaut sein kann
und was konkret die Herausforderungen sind, die sie bewältigen muss,
können Sie im KITcar Teil noch mehr erfahren.
